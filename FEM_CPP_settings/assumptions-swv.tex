\documentclass{article}
\usepackage{savetrees}
\usepackage{hyperref}

\title{Assumptions within the FEM}

\begin{document}
\maketitle

\begin{Rcode}{!echo,hide}
library(ggplot2)
\end{Rcode}

\section{Economic Factors}

Inflation halts in 2010, so all growth rates afterwards are real
growth.  The following are represented:
\begin{itemize}
\item Interest Rate on Wealth
\item Earning Disregard
\item National Wage Index (SA 2009 TR Intermediate Cost Assumptions,
  Table VI.F6)
\item Social Security Wage Cap
\item Medicare Cost Growth (CMS 2009 Trustees Report, p179)
\item Gross Domestic Product
\begin{description}
\item[1992-2009] Historic
\item[2010-2019] CBO analysis of President's Budget
\item[2020-2100] 2008 OASDI Trustees Report 2.1\%
\item[Labor Force] 2008 OASDI Trustees Report
\end{description}
\item Total Medicare Cost Growth (Computed)
\end{itemize}

\subsection{Interest Rate on Wealth}
This rate is used to increase household wealth year-over-year. It is a
real interest rate over and above CPI.

\begin{Rcode}{!echo,fig,savefig,label=interest}
interest <- read.table('timeseries/interest_rate.txt', skip=1)
names(interest) <- c('year','cummulative.interest')
ggplot(interest, aes(x=year, y=cummulative.interest)) + geom_line()
\end{Rcode}

\begin{figure}[h]
\centering
\recallfig{interest}
\caption{Cummulative Real Interest Rate}
\label{fig:interest}
\end{figure}

\subsection{Earning Disregard}
Below full retirement age, \$1 is deducted from SS benefits for every
\$2 earned above the limit. Above full retirement age, \$1 is deducted
for every \$3 earned above the limit.

\begin{Rcode}{!echo,fig,savefig,label=earndisreg}
disreg1 <- read.table('timeseries/eadisreg1.txt', skip=1)
disreg2 <- read.table('timeseries/eadisreg2.txt', skip=1)
names(disreg1) <- c('year','Earning.Disregard')
names(disreg2) <- names(disreg1)
disreg1$group <- 'before retirement'
disreg2$group <- 'after retirement'
disreg <- rbind(disreg1, disreg2)
disreg$Earning.Disregard <- disreg$Earning.Disregard/1e3
ggplot(disreg, aes(x=year, y=Earning.Disregard, colour=group)) +geom_line()
\end{Rcode}

\begin{figure}[h]
\centering
\recallfig{earndisreg}
\caption{Earning Disregard (Thousands) Projected 1.1\% Real Growth}
\label{fig:earndisreg}
\end{figure}

\subsection{National Wage Index}
The National Wage Index is used to calculate or update a number of quantities year
over year:
\begin{itemize}
\item AIME
\item Earnings
\item Principle Insurance Amount
\item Widow's Benefit
\end{itemize}

\begin{Rcode}{!echo,fig,savefig,label=nwi}
nwi <- read.table('timeseries/nwi.txt', skip=1)
nwi$V2 <- nwi$V2/1e3
names(nwi) <- c('year','NWI')
ggplot(nwi, aes(x=year, y=NWI)) + geom_line()
\end{Rcode}

\begin{figure}[h]
\centering
\recallfig{nwi}
\caption{National Wage Index in Thousands}
\label{fig:nwi}
\end{figure}

\subsection{Social Security Wage Cap}
This is the maximum amount of income subject to the Social Security
payroll tax. It is only used in computing taxes.

\begin{Rcode}{!echo,fig,savefig,label=sscap}
sscap <- read.table('timeseries/sscap.txt', skip=1)
names(sscap) <- c('year', 'SS.Cap')
sscap$SS.Cap <- sscap$SS.Cap / 1e3
ggplot(sscap, aes(x=year, y=SS.Cap)) + geom_line()
\end{Rcode}

\begin{figure}[h]
\centering
\recallfig{sscap}
\caption{Social Security Wage Cap (Thousands)}
\label{fig:sscap}
\end{figure}

\subsection{Total Medicare Cost Growth}
This is the projected growth in overall Medicare costs and is used to
inflate medicaid and medicare costs. This is added on top of
per-laborer GDP growth and then divided by the growth in the enrollee
pool (simulated) to figure out the per-enrollee cost growth.

\begin{Rcode}{!echo,fig,savefig,label=cummulative}
c.med <- read.table('timeseries/medgrowth.txt',skip=1)
c.gdp <- read.table('timeseries/gdp.txt', skip=4)
c.labor <- read.table('timeseries/labor.txt', skip=2)
names(c.med) <- c('year','Growth')
names(c.gdp) <- c('year','Growth')
names(c.labor) <- c('year','Growth')

c.med[,'series'] <- 'Medicare Cost above GDP'
c.gdp[,'series'] <- 'Real GDP'
c.labor[,'series'] <- 'Labor Force'

row.names(c.med) <- c.med[,'year']
row.names(c.gdp) <- c.gdp[,'year']
row.names(c.labor) <- c.labor[,'year']

c.percap <- c.gdp
c.percap[,'Growth'] <- c.gdp[,'Growth']/c.labor[,'Growth'] * 100
c.percap[,'series'] <- 'GDP Per Labor Force'

c.totmed <- c.med
c.totmed[,'series'] <- 'Total Medicare Cost'
c.totmed[,'Growth'] <- c.percap[,'Growth'] * c.med[,'Growth'] / 100

a.med <- c.med[2:nrow(c.med),]
a.gdp <- c.gdp[2:nrow(c.gdp),]
a.labor <- c.labor[2:nrow(c.labor),]
a.percap <- c.percap[2:nrow(c.percap),]
a.totmed <- c.totmed[2:nrow(c.totmed),]

c.med[,'group'] <- 'Cummulative'
c.gdp[,'group'] <- 'Cummulative'
c.labor[,'group'] <- 'Cummulative'
c.percap[,'group'] <- 'Cummulative'
c.totmed[,'group'] <- 'Cummulative'
a.med[,'group'] <- 'Annualized'
a.gdp[,'group'] <- 'Annualized'
a.labor[,'group'] <- 'Annualized'
a.percap[,'group'] <- 'Annualized'
a.totmed[,'group'] <- 'Annualized'

med.num <- as.character(c.med[2:nrow(c.med),'year'])
med.den <- as.character(c.med[1:(nrow(c.med)-1),'year'])
gdp.num <- as.character(c.gdp[2:nrow(c.gdp),'year'])
gdp.den <- as.character(c.gdp[1:(nrow(c.gdp)-1),'year'])
labor.num <- as.character(c.labor[2:nrow(c.labor),'year'])
labor.den <- as.character(c.labor[1:(nrow(c.labor)-1),'year'])
percap.num <- as.character(c.percap[2:nrow(c.percap),'year'])
percap.den <- as.character(c.percap[1:(nrow(c.percap)-1),'year'])

a.med[,'Growth'] <- c.med[med.num,'Growth'] / c.med[med.den,'Growth']
a.gdp[,'Growth'] <- c.gdp[gdp.num,'Growth'] / c.gdp[gdp.den,'Growth']
a.labor[,'Growth'] <- c.labor[labor.num,'Growth'] / c.labor[labor.den,'Growth']
a.percap[,'Growth'] <- c.percap[percap.num,'Growth'] /c.percap[percap.den,'Growth']
a.totmed[,'Growth'] <- c.totmed[med.num,'Growth'] /c.totmed[med.den,'Growth']

f <- rbind(c.gdp, c.med, c.labor, c.percap, c.totmed)
ggplot(f) + geom_line(aes(x=year, y=Growth, color=series)) + opts(legend.position = "bottom")
\end{Rcode}

\begin{figure}[h]
\centering
\recallfig{cummulative}
\caption{Cummulative Medicare Cost Growth Breakdown}
\label{fig:cummulative}
\end{figure}

\begin{Rcode}{!echo,fig,savefig,label=annualized}
f <- rbind(a.gdp, a.med, a.labor, a.percap, a.totmed)
ggplot(f) + geom_line(aes(x=year, y=Growth, color=series)) + opts(legend.position = "bottom")
\end{Rcode}

\begin{figure}[h]
\centering
\recallfig{annualized}
\caption{Annualized Medicare Cost Growth Breakdown}
\label{fig:annualized}
\end{figure}

\end{document}