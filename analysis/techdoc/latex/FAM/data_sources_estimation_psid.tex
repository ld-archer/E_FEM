\subsection{Panel Survey of Income Dynamics}
The Panel Survey of Income Dynamics (PSID), waves 1999-2013 are used to estimate the transition models. 
PSID interviews occur every two years.  We create a dataset of respondents who have formed their own households, either
as single heads of households, cohabitating partners, or married partners.  These heads, wives, and "wives" (males
are automatically assigned head of household status by the PSID if they are in a couple) respond to the richest
set of PSID questions, including the health questions that are critical for our purposes.

We use all respondents age 25 and older.  When appropriately weighted, the PSID is representative of U.S. households.  
We also use the PSID as the host data for full population simulations that begin in 2009.  Respondents age 25 and 26 
are used as the basis for the synthetic cohorts that we generate, used for replenishing the sample in population 
simulations or as the basis of cohort scenarios.  

The PSID continually adds new cohorts that are descendents (or new partners/spouses of descendents).  Consequently,
updating the simulation to include more recent data is straightforward.