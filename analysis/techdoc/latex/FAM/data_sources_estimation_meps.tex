\subsection{Medical Expenditure Panel Survey}
\label{sec:data_sources_estimation_meps}
The Medical Expenditure Panel Survey (MEPS), beginning in 1996, is a set of large-scale surveys of 
families and individuals, their medical providers (doctors, hospitals, pharmacies, etc.), and 
employers across the United States. The Household Component (HC) of the MEPS provides data from 
individual households and their members, which is supplemented by data from their medical providers. 
The Household Component collects data from a representative sub sample of households drawn from the 
previous year's National Health Interview Survey (NHIS). Since NHIS does not include the 
institutionalized population, neither does MEPS: this implies that we can only use the MEPS to 
estimate medical costs for the non-elderly (25-64) population. Information collected during household 
interviews include: demographic characteristics, health conditions, health status, use of medical 
services, sources of medical payments, and body weight and height. Each year the household survey 
includes approximately 12,000 households or 34,000 individuals. Sample size for those aged 25-64 is 
about 15,800 in each year.  MEPS has comparable measures of social-economic (SES) variables as those in PSID, 
including age, race/ethnicity, educational level, census region, and marital status.  We estimate expenditures 
and utilization using 2007-2010 data.

See Section 
\ref{sec:govt_revenue_and_expenditures_medcost_estimation} for a description.  FAM also
uses MEPS 2001-2003 data for QALY model estimation. 

% This is described in Section  \ref{sec:estimation_qalys_crosswalk_development}.

