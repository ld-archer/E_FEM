\subsection{Cross-validation}
The cross-validation exercise randomly samples half of the PSID respondent IDs for use in estimating the transition models.  The respondents not used for estimation, 
but who were present in the PSID sample in 1999, are then simulated from 1999 through 2013.  Demographic, health, and economic outcomes are compared between the 
simulated (�FAM�) and actual (�PSID�) populations.  

%These results are presented in Table \ref{tab:crossval_unweighted} - Table \ref{tab:crossval_cntecon} for 2000, 2006, and 2012, 
%with a statistical test of the difference between the average values in the two populations.

Worth noting is how the composition of the population changes in this exercise.  In 1999, the sample represents those 25 and older.  Since we follow a fixed cohort, 
the age of the population will increase to 39 and older in 2013.  This has consequences for some measures in later years where the eligible population shrinks.

\subsubsection{Demographics}
Mortality and demographic measures are presented in Tables \ref{tab:crossval_unweighted} and \ref{tab:crossval_demog}.  Mortality incidence is comparable between
the simulated and observed populations.  Demographic characteristics do not differ between the two.

\subsubsection{Health Outcomes}
Binary health outcomes are presented in Table \ref{tab:crossval_binhlth}. FAM underestimates the prevalence of ADL and IADL limitations compared to the crossvalidation sample. 
Binary outcomes, like cancer, diabetes, heart diesease, and stroke do not differ.  FAM underpredicts hypertension and lung disease compared to the crossvalidation sample.

\subsubsection{Health Risk Factors}
Risk factors are presented in Table \ref{tab:crossval_risk}.  BMI is not statistically different between the two samples.  Current smoking is not statistically different,
 but more individuals in the crossvalidation sample report being former smokers.

\subsubsection{Economic Outcomes}
Binary economic outcomes are presented in Table \ref{tab:crossval_binecon}.  FAM underpredicts claiming of federal disability and overpredicts Social Security retirement 
claiming.  Supplemental Security claiming is not statistically different between FAM and the crossvalidation sample.  Working for pay is not statistically different.

%Continuout economic outcomes are presented in Table \ref{tab:crossval_cntecon}.

On the whole, the crossvalidation exercise is reassuring.  There are differences that will be explored and improved upon in the future.
