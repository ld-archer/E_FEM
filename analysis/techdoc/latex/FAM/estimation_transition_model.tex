\subsection{Transition model}
\label{sec:estimation_transition_model} 
We consider a large set of outcomes for which we model transitions. Table 
\ref{tab:transitioned_outcomes} gives the set of outcomes considered for the 
transition model along with descriptive statistics and the population at risk 
when estimating the relationships. 

Since we have a stock sample from the age 25+ population, each respondent goes 
through an individual-specific series of intervals. Hence, we have an 
unbalanced panel over the age range starting from 25 years old. Denote by 
$j_{i0}$ the first age at which respondent $i$ is observed and $j_{iT_i}$ the 
last age when he is observed. Hence we observe outcomes at ages 
$j_i = j_{i0},\ldots,j_{iT_i}$. 

We first start with discrete outcomes which are absorbing states (e.g. disease 
diagnostic, mortality, benefit claiming). Record as $h_{i,j_i,m}=1$ if the 
individual outcome $m$ has occurred as of age $j_i$. We assume the 
individual-specific component of the hazard can be decomposed in a time 
invariant and variant part. The time invariant part is composed of the effect of 
observed characteristics $x_i$ that are constant over the entire life course and 
initial conditions $h_{i,j_0,-m}$ (outcomes other than the outcome 
$m$) that are determined before the first age in which each individual is observed
%\footnote{Section \ref{sec:model_development_transition_model} explains why the $h_{i,j_0,-m}$ terms are included.}. 
The time-varying part is the effect of previously diagnosed outcomes $h_{i,j_i-1,-m}$, 
on the hazard for $m$.\footnote{With some abuse of notation, $j_i-1$ denotes 
the previous age at which the respondent was observed.}  We assume an index of 
the form  $z_{m,j_i} = x_i\beta_m + h_{i,j_i-1,-m} \gamma_m + h_{i,j_0,-m}\psi_m$. Hence, the 
latent component of the hazard is modeled as 
\begin{equation}
h^*_{i,j_i,m}= x_i\beta_m + h_{i,j_i-1,-m} \gamma_m + h_{i,j_0,-m}\psi_m + a_{m,j_i} + \varepsilon_{i,j_i,m},
\label{eqn:transition_hzd_latent}
\end{equation}
\[
m = 1,\ldots,M_0 \mbox{, } j_i = j_{i0},\ldots,j_{i,T_i} \mbox{, } i=1,\ldots,N
\]
The term $\varepsilon_{i,j_i,m}$ is a time-varying shock specific to age $j_i$. 
We assume that this last shock is normally distributed and uncorrelated across 
diseases.  We approximate $a_{m,j_i}$ with an age spline with knots at ages 35, 45,
55, 65, and 75.
This simplification is made 
for computational reasons since the joint estimation with unrestricted age fixed 
effects for each condition would imply a large number of parameters.  The 
absorbing outcome, conditional on being at risk, is defined as 
\[
h_{i,j_i,m} = \max\{I(h^*_{i,j_i,m} > 0), h_{i,j_i-1,m}\}
%\label{eqn:transition_outcome}
\]
The occurrence of mortality censors observation of other 
outcomes in a current year. 
% Mortality is recorded from exit interviews.

A number of restrictions are placed on the way feedback is allowed in the model.  
Table \ref{tab:transition_restrictions} documents restrictions placed on the 
transition model. We also include a set of other controls. A list of such controls 
is given in Table \ref{tab:controlvar_descs} along with descriptive statistics. 

We have five other types of outcomes:
\begin{enumerate}
\item First, we have binary outcomes which are not an absorbing state, such as starting smoking. We specify latent 
indices as in (\ref{eqn:transition_hzd_latent}) for these outcomes as well but where 
the lag dependent outcome also appears as a right-hand side variable. This allows for state-dependence. 

\item Second, we have ordered outcomes. These outcomes are also modeled as in (\ref{eqn:transition_hzd_latent}) recognizing the observation rule is 
a function of unknown thresholds $\varsigma_m$. Similarly to binary outcomes, we allow for state-dependence by including the lagged outcome 
on the right-hand side.

\item The third type of outcomes we consider are censored outcomes, such as financial wealth. For wealth, 
there are a non-negligible number of observations with zero and negative wealth. For these, we consider 
two part models where the latent variable is specified as in (\ref{eqn:transition_hzd_latent}) but model 
probabilities only when censoring does not occur. In total, we have $M$ outcomes.

\item The fourth type of outcomes are continuous outcomes modeled with ordinary least squares.  For example,
we model transitions in log(BMI).  We allow for state-dependence by including the lagged outcome on the right-hand side.

\item The final type of models are categorical, but without an ordering.  For example, an individual can transition
to being out of the labor force, unemployed, or working (either full- or part-time).  In situations like this, we utilize
a multinomial logit model, including the lagged outcome on the right-hand side.

\end{enumerate}

The parameters 
$\mathbf{\theta}_1 = \left(\left\{\beta_m, \gamma_m, \psi_m, \varsigma_m\right\}_{m=1}^M, \right)$, 
can be estimated by maximum likelihood. Given the normality distribution assumption on the 
time-varying unobservable, the joint probability of all time-intervals until failure, right-censoring 
or death conditional on the initial conditions $h_{i,j_0,-m}$ is the product of 
normal univariate probabilities. Since these sequences, conditional on initial 
conditions, are also independent across diseases, the joint 
probability over all disease-specific sequences is simply the product of 
those probabilities. 

For a given respondent observed from initial age $j_{i0}$ to a last age $j_{T_i}$, the probability of the observed health history is 
(omitting the conditioning on covariates for notational simplicity)
\[
	l^{-0}_i(\mathbf{\theta}; h_{i,j_{i0}}) = \left[\prod_{m=1}^{M-1} \prod_{j=j_{i1}}^{j_{T_i}} P_{ij,m}(\mathbf{\theta})^{(1-h_{ij-1,m})(1-h_{ij,M})} \right] \times \left[\prod_{j=j_{i1}}^{j_{T_i}} P_{ij,M}(\mathbf{\theta}) \right]
\]
We use the ${-0}$ superscript to make explicit the conditioning on $\mathbf{h}_{i,j_{i0}} = (h_{i,j_{i0},0},\ldots,h_{i,j_{i0},M})'$. We have limited information on outcomes prior to this age. 
The likelihood is a product of $M$ terms with the $m$th term containing only 
$(\beta_m, \gamma_m, \psi_m, \varsigma_m)$.  This allows the estimation
to be done seperately for each outcome.

\subsubsection{Further Details on Specific Transition Models}
This section describes the modeling strategy for particular outcomes.

\paragraph{Employment Status}
Ultimately, we wish to simulate if an individual is out of the labor force, unemployed, working part-time, or working full-time at 
time $t$.  We treat the estimation of this as a two-stage process.  In the first stage, we predict if the individual is out of 
the labor force, unemployed, or working for pay using a multinomial logit model.  Then, conditional on working for pay, we estimate if 
the individual is working part- or full-time using a probit model.

\paragraph{Earnings}
We estimate last calendar year earnings models based on the current employment status, controlling for the prior employment status.  
Of particular concern are individuals with no earnings, representing approximately twenty-five percent of the unemployed and 
seventy-eight percent of those out of the labor force.  This group is less than 0.5\% of the full- and part-time populations. We 
use a two-stage process for those out of the labor force and unemployed.  The first stage is a probit that estimates if the 
individual has any earnings.  The second stage is an OLS model of log(earnings) for those with non-zero earnings.  For 
those working full- or part-time, we estimate OLS models of log(earnings). 

\paragraph{Relationship Status}
We are interested in three relationship statuses: single, cohabitating, and married.  In each case, we treat the transition
from time $t$ to time $t+1$ as a two-stage process.  In the first stage, we estimate if the individual will remain in their 
current status.  In the second stage, we estimate which of the two other states the individual will transition to, conditional
on leaving their current state.

\paragraph{Childbearing}
We estimate the number of children born in two-years separately for women and men.  We model this using an ordered probit with
three categories: no new births, one birth, and two births.  Based on the PSID data, we found the exclusion of three or more
births in a two-year period to be appropriate.  

\subsubsection{Inverse Hyperbolic Sine Transformation}
One problem fitting the wealth distribution is that it has a long right tail and some negative values. We use a 
generalization of the inverse hyperbolic sine transform (IHT) presented in \citet{mackinnon1990transforming}. First denote the variable of 
interest $y$. The hyperbolic sine transform is 
\begin{equation}
y = \sinh(x) = \frac{\exp(x) - \exp(-x)}{2}
\label{eqn:sinh_y}
\end{equation}
The inverse of the hyperbolic sine transform is
\[
x = \sinh^{-1}(y) = h(y) = \log(y + (1+y^2)^{1/2})
\]
Consider the inverse transformation. We can generalize such transformation, first allowing for a 
shape parameter $\theta$,
\begin{equation}
r(y) = h(\theta y)/\theta
\label{eqn:generalized_ihs_shape}
\end{equation}
Such that we can specify the regression model as
\begin{equation}
r(y) = x\beta + \varepsilon, \varepsilon \sim \mathrm{N}(0, \sigma^2)
\label{eqn:ihs_regression_model}
\end{equation}
A further generalization is to introduce a location parameter $\omega$ such that the new 
transformation becomes
\begin{equation}
g(y) = \frac{h(\theta(y+\omega)) - h(\theta\omega)}{\theta h'(\theta \omega)}
\label{eqn:geralized_ihs_loc_scale}
\end{equation}
where $h'(a) = (1+a^2)^{-1/2}$. 

We specify (\ref{eqn:ihs_regression_model}) in terms of the transformation $g$. The shape parameters 
can be estimated from the concentrated likelihood for $\theta, \omega$. We can then 
retrieve $\beta, \sigma$ by standard OLS. 

Upon estimation, we can simulate 
\[
\tilde{g} = x \hat{\beta} + \sigma \tilde{\eta}
\]
where $\eta$ is a standard normal draw. Given this draw, we can retransform using 
(\ref{eqn:geralized_ihs_loc_scale}) and (\ref{eqn:sinh_y})
\begin{align*}
&h(\theta(y+\omega)) = \theta h'(\theta\omega)\tilde{g} + h(\theta\omega)\\
&\tilde{y} = \frac{\sinh\left[\theta h'(\theta\omega)\tilde{g} + h(\theta\omega)\right]-\theta\omega}{\theta}
\end{align*}

% \todo link to transition_estimates.xls on box -- make link unique to the version of the appendix in make process
The included estimates table (estimates\_FAM.xml) gives parameter estimates for the transition models.

