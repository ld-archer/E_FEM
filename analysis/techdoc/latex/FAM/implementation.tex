\section{Implementation}
The FAM is implemented in multiple parts. Estimation of the transition and cross 
sectional models is performed in Stata.
The replenishing cohort model is estimated in Stata using the CMP package \citep{statacmp2011}.
The simulation is implemented in C++ for speed and flexibility.  Currently, the simulation is
run on Linux, Windows, and Mac OS X.  

To match the two year structure of the PSID data used to estimate the 
transition models, the FAM simulation proceeds in two year increments. The end 
of each two year step is designed to occur on July 1st to allow for easier 
matching to population forecasts from Social Security. A simulation of the FAM 
proceeds by first loading a population representative of the age 25+ US 
population in 2009, generated from PSID. In two year increments, the FAM applies 
the transition models for mortality, health, working, wealth, earnings, and 
benefit claiming with Monte Carlo decisions to calculate the new states of the 
population. The population is also adjusted by immigration forecasts from the 
US Census Department, stratified by race and age. If incoming cohorts are being 
used, the new 25/26 year olds are added to the population. The number of new 
25/26 year olds added is consistent with estimates from the Census, stratified 
by race. Once the new states have been determined and new 25/26 year olds added, 
the cross sectional models for medical costs are performed. Summary variables 
are then computed. Computation of medical costs includes the persons that died 
to account for end of life costs. To reduce uncertainty due to the Monte 
Carlo decision rules, the simulation is performed multiple times (typically 
100), and the mean of each summary variable is calculated across repetitions. 

FAM simulation takes as inputs assumptions regarding  
% [excluding for now...]
%the national wage index, 
the normal retirement age, real medical cost growth, and interest rates. The 
default assumptions are taken from the 2010 Social Security Intermediate 
scenario, adjusted for no price increases after 2010. Therefore simulation 
results are in real 2009 dollars.  
%Table \ref{tab:time_series_by_year} shows the 
%assumptions for each calendar year and Table \ref{tab:time_series_by_yob} shows 
%assumptions for each birth year.

Different simulation scenarios are implemented by changing any of the following 
components: incoming cohort model, transition models, interventions that adjust 
the probabilities of specific transition, and changes to assumptions on future 
economic conditions.  	

\subsection{Intervention Module}
The intervention module can adjust characteristics of individuals when they are first read into the simulation ``init\_interventions'' or alter transitions within the 
simulation ``interventions.''  At present, init\_interventions can act on chronic diseases, ADL$/$IADL status, program participation, and some demographic characteristics.  
Interventions within the simulation can currently act on mortality, chronic diseases, and some program participation variables.

Interventions can take several forms.  The most commonly used is an adjustment to a transition probability. One can also delay the assignment of a chronic condition or cure an existing 
chronic condition.  Additional flexibility comes from selecting who is eligible for the intervention.  Some examples might help to make the interventions concrete.

\begin{itemize}
\item Example 1: Delay the enrollment into Social Security OASI by two years.  In this scenario claiming of Social Security benefits is transitioned as normal.  However, if a person is predicted to 
claim their benefits, then that status is not immediately assigned, but is instead assigned two years later.

\item Example 2: Cure hypertension for those with no other chronic diseases.  In this scenario any individual with hypertension (including those who have had hypertension for many years) is cured 
(hypertension status is set to 0), as long as they do not have other chronic diseases.  This example uses the individual�s chronic disease status as the eligibility criteria for the intervention.

\item Example 3: Reduce the incidence of hypertension for half of men aged 55 to 65 by 10\% in the first year of the simulation, gradually increasing the reduction to 20\% after 10 years.  This 
example begins to show the flexibility in the intervention module.  The eligibility criteria are more complex (half of men in a specific age range are eligible) and the intervention 
changes over time.  Mathematically, the intervention works by acting on the incidence probability, $\rho$. In the first year of the simulation, the probability is replaced by 
 $\left(1-0.5*0.1\right)\rho=0.95\rho$.  The binary outcome is then assigned based on this new probability.  Thus, at the population level, there is a 5\% reduction in incidence for men aged 55 to 65, 
as desired.  After 10 years, the probability for this eligible population becomes $\left(1-0.5*0.2\right)\rho=0.9\rho$.
\end{itemize}

More elaborate interventions can be programmed by the user.  



