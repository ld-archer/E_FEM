\subsection{Comparison with other microsimulation models of health expenditures}
The precursor to the FAM, the FEM, was unique among models that make health expenditure projections.  It was the only 
model that projected health trends rather than health expenditures.  It was also unique in generating mortality projections
based on assumptions about health trends rather than historical time series.

FAM extends FEM to younger ages, adding additional dimensions to the simulation.  Events over the life course, such as marital status
and childbearing are simulated.  Labor force participation is modeled in greater detail, distinguishing between out-of-labor force, unemployed, 
working part-time, and working full-time.

\subsubsection{CBOLT Model}
The Congressional Budget Office (CBO) uses time-series techniques to project health expenditure 
growth in the short term and then makes an assumption on long-term growth. They use a long term 
growth of excess costs of 2.3 percentage points starting in 2020 for Medicare. They then assume a 
reduction in excess cost growth in Medicare of 1.5\% through 2083, leaving a rate of 0.9\% in 2083. 
For non-Medicare spending they assume an annual decline of 4.5\%, leading to an excess growth rate in 
2083 of 0.1\%. 

\subsubsection{Centers for Medicare and Medicaid Services}
The Centers for Medicare and Medicaid Services (CMS) performs an extrapolation of medical expenditures 
over the first ten years, then computes a general equilibrium model for years 25 through 75 and 
linearly interpolates to identify medical expenditures in years 11 through 24 of their estimation. 
The core assumption they use is that excess growth of health expenditures will be one percentage point 
higher per year for years 25-75 (that is if nominal GDP growth is 4\%, health care expenditure growth 
will be 5\%).

\subsubsection{MINT Model}
Modeling Income in the Near Term (MINT) is a microsimulation model developed by the Urban Institute and others 
for the Social Security Administration to enable policy analysis of proposed changes to Social Security benefits and payroll
taxes \citet{smith2013mint}.  MINT uses the Survey of Income and Program Participation (SIPP) as the base data and simulates a range of outcomes, 
with a focus on those that will impact Social Security.  Recent extensions have included health insurance coverage and
out-of-pocket medical expenditures.  Health enters MINT via self-reported health status and self-reported work limitations.  MINT
simulates marital status and fertility. 

