\subsection{Medicare Current Beneficiary Survey}
The Medicare Current Beneficiary Survey (MCBS) is a nationally representative sample of aged, disabled 
and institutionalized Medicare beneficiaries.  The MCBS attempts to interview each respondent twelve 
times over three years, regardless of whether he or she resides in the community, a facility, or 
transitions between community and facility settings. The disabled (under 65 years of age) and 
oldest-old (85 years of age or older) are over-sampled. The first round of interviewing was conducted 
in 1991. Originally, the survey was a longitudinal sample with periodic supplements and indefinite 
periods of participation. In 1994, the MCBS switched to a rotating panel design with limited periods 
of participation. Each fall a new panel is introduced, with a target sample size of 12,000 respondents 
and each summer a panel is retired. Institutionalized respondents are interviewed by proxy.  The MCBS 
contains comprehensive self-reported information on the health status, health care use and 
expenditures, health insurance coverage, and socioeconomic and demographic characteristics of the 
entire spectrum of Medicare beneficiaries.  Medicare claims data for beneficiaries enrolled in 
fee-for-service plans are also used to provide more accurate information on health care use and 
expenditures.  MCBS years 2007-2010 are used for estimating medical cost and enrollment models.
See section \ref{sec:govt_revenue_and_expenditures_medcost_estimation} for discussion.
