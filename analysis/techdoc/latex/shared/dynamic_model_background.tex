\subsection{Background}
The Future Elderly Model (FEM) is a microsimulation model originally developed out of an effort to 
examine health and health care costs among the elderly Medicare population (age 65+). A description 
of the previous incarnation of the model can be found in \citet{goldman2004health}. The original work was 
founded by the Centers for Medicare and Medicaid Services and carried out by a team of researchers 
composed of Dana P. Goldman, Paul G. Shekelle, Jayanta Bhattacharya, Michael Hurd, Geoffrey F. Joyce, 
Darius N. Lakdawalla, Dawn H. Matsui, Sydne J. Newberry, Constantijn W. A. Panis and Baoping Shang.

Since then various extensions have been implemented to the original model. The most recent version 
now projects health outcomes for all Americans aged 51 and older and uses the Health and Retirement 
Study (HRS) as a host dataset rather than the Medicare Current Beneficiary Survey (MCBS).  The work 
has also been extended to include economic outcomes such as earnings, labor force participation and 
pensions. This work was funded by the National Institute on Aging through its support of the RAND 
Roybal Center for Health Policy Simulation (P30AG024968), the Department of Labor through contract 
J-9-P-2-0033, the National Institutes of Aging through the R01 grant ``Integrated Retirement 
Modeling'' (R01AG030824) and the MacArthur Foundation Research Network on an Aging Society. Finally, 
the computer code of the model was transferred from Stata to C++. This report incorporates these new 
development efforts in the description of the model.
% \todo add description of work extending to ages 25+ (FAM) here
