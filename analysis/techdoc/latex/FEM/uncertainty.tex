\section{Uncertainty}
Validation of population-based simulation models includes uncertainty analysis (Kopec 2010). Uncertainty analysis 
is the "quantification of the differences between a model's estimates and the truth." (Citro and Hanushek 1991). Sources of 
uncertainty common in population-based microsimulation models include sampling variability from input databases, sampling 
variability from other inputs, model misspecification, and stochastic error (Citro and Hanushek 1991). Stochastic and sampling 
uncertainty can be evaluated using confidence intervals around the output point estimates.  

\subsection{Confidence intervals}
The FEM produces output parameters for each repetition of the simulation. A confidence interval can be calculated from the sample 
of output parameters when the input parameters are varied with each repetition. Confidence intervals from the deterministic simulation 
only take into account the intrinsic error from the Monte Carlo stochastic element of the simulation. Repeating the deterministic estimation 
for 50 to 100 replicates is considered to reduce the Monte Carlo error to nearly zero. Stochastic uncertainty alone does not adequately 
account for uncertainty in microsimulation results.

\subsection{Bootstraping}
Sampling variability from the survey sources is considered, in the FEM, under the conditions of a resampled survey. Direct 
bootstrapping is used as a conservative technique to account for the other sources of uncertainty. The data is randomly sampled with replacement 
taking into account survey design (Yeo, H and Liu 1999). The HRS is sampled at the household level with 56 strata in recent waves. 
Observations are selected for their full panel of interviews. Each input is produced for each sample of the surveys. Bootstrapping of 
the inputs implicitly takes into account the correlation between the parameters used in the FEM, particularly from the HRS. We use dependent 
bootstrap samples for all sources of sampling uncertainty from the HRS: input database, transition parameters, and trends.

The number of bootstrap samples is started at 1000 based on what is found in the literature, but the total number used depends on the outcome 
and models being assessed. Watching the variance of the output parameter to see where it settles down offers an estimate of the number of 
repetitions needed. Each rep then has a mean and non-parametric distribution from which the 2.5 and 97.5 percentiles can be used as confidence 
intervals. The mean is calculated as the mean over all observations (Dong and Nakayama 2014). The confidence intervals are calculated as the 
mean of the upper and lower percentiles. The contributions of each source of uncertainty discussed in more detail below can be estimated by 
independently combining deterministic and sampled inputs when running the simulation. 

\subsubsection{Sample uncertainty}
In the FEM, the major sources of input uncertainty come from sampling variability. The FEM uses a single wave from the HRS as the population that 
starts the simulation. In order to incorporate the sampling uncertainty in the FEm, bootstrap samples are taken from the population prior to 
population reweighting. Each bootstrap sample is reweighted to the population  before entering the simulation. 

\subsubsection{Trends}
Trends are used to alter the health and economic characteristics of future cohorts before they enter the simulation are estimated from survey data or 
pulled from the literature. Because distributions of the trend parameters in each year are not available from all sources, we use 
bootstrap samples of each cohort after applying the baseline trends in order to achieve variation in the trended parameters. 

\subsubsection{Transition parameter uncertainty}
Transition probabilities in the FEM are predicted for each active individual in each wave of the simulation using regression models 
estimated from the HRS. Regression models for transition probabilities are estimated using the bootstrap samples of the HRS which 
implicitly allow for correlation of covariates within models and between models. A set of transition regression models is estimated 
for each bootstrap sample.

\subsubsection{Models that predict policy outcomes}
The regression models that predict cost, utilization, and other outcomes are based on MCBS and MEPS. These data are also subject to 
sampling error and are sampled in a similar way to the HRS. The MCBS is sampled at the individual level with 100 strata. The MEPS 
is sampled at the dwelling unit level with 165 strata in recent surveys. The MEPS and MCBS 
bootsrap weights are used in the estimation of the medical spending and utilization regression models.
