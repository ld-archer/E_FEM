\subsection{Information available and empirical strategy}
\label{sec:new_cohorts_model_empirical_strategy}
For the transition model, we need to first to obtain outcomes listed in Table \ref{tab:desc_init_conditions}. Ideally, we need 
information on 
\[
f_t(y_{i1},\ldots,y_{iM}) = f_t(\mathbf{y}_i)
\]
where $t$ denotes calendar time, and $\mathbf{y}_i = (y_{i1},\ldots,y_{iM})$ is a vector of outcomes 
of interest whose probability distribution at time $t$ is $f_t()$. Information on how the joint 
distribution evolves over time is not available. Trends in conditional distributions are rarely 
reported either.

Generally, we have (from published or unpublished sources) good information on trends for some moments 
of each outcome (say a mean or a fraction). That is, we have information on 
$g_{t,m}(y_{im})$,
where $g_{t,m}()$ denotes the marginal probability distribution of outcome $m$ at time $t$. 

For example, we know from the NHIS repeated cross-sections that the fraction obese is increasing by 
roughly 2\% a year among 51 year olds. In statistical jargon this means we have information on how the 
mean of the marginal distribution of $y_{im}$, an indicator variable that denotes whether someone is obese, 
is evolving over time. 

We also have information on the joint distribution at one point in time, say year $t_0$. For example,
 we can estimate the joint distribution on age 51 respondents in the 1992 wave of the HRS, $f_{t_0}(\mathbf{y}_i)$. 

We make the assumption that only some part of $f_t(\mathbf{y}_i)$ evolves over time. In particular, 
we will model the marginal distribution of each outcome allowing for correlation across these 
marginals. The correlations will be assumed fixed while the mean of the marginals will be allowed to 
change over time. 
