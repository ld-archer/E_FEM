\subsection{Cross-validation}
The cross-validation exercise randomly samples half of the HRS respondent IDs for use in estimating the transition models.  The respondents not used for estimation, 
but who were present in the HRS sample in 1998, are then simulated from 1998 through 2012.  Demographic, health, and economic outcomes are compared between the 
simulated (�FEM�) and actual (�HRS�) populations.  These results are presented in Table \ref{tab:crossval_unweighted} - Table \ref{tab:crossval_cntecon} for 2000, 2006, and 2012, 
with a statistical test of the difference between the average values in the two populations.

Worth noting is how the composition of the population changes in this exercise.  In 1998, the sample represents those 51 and older.  Since we follow a fixed cohort, 
the age of the population will increase to 65 and older in 2012.  This has consequences for some measures in later years where the eligible population shrinks.

\subsubsection{Demographics}
Demographic measures are presented in Table \ref{tab:crossval_demog}.  Demographic differences between the two populations are small.  The gender balance and fraction 
of the population that is non-Hispanic Black or Hispanic is consistent. 

\subsubsection{Health Outcomes}
The FEM population has a slightly higher population with one or more ADL limitation in 2012 (19.6% vs 17.5%).  Those with any IADL limitations are not statistically 
different from one another in 2012.

The two populations are not statistically different from each other for prevalence of cancer, hypertension, or lung disease in 2012.  They do differ for diabetes 
(26.8% for the FEM, 25.0% for the HRS), heart disease (35.3% for the FEM, 32.8% for the HRS), and stroke (12.3% for FEM, 11.1% for the HRS), though the practical 
significance of these differences is not clear.

\subsubsection{Health Risk Factors}
Average BMI is slightly higher for the FEM population in 2012 (28.0 for the FEM vs. 27.7 for the HRS).  In terms of practical significance, this difference is 
equivalent to two pounds for an individual who is 5�8�.  Smoking behavior is not statistically different between the two populations in 2012.  The nursing home 
population is also not statistically different between the FEM and the HRS.

\subsubsection{Economic Outcomes}
Social Security claiming and Supplemental Security Income claiming do not differ between the FEM and the HRS by 2012.  The population working is slightly lower 
in the FEM (22.9% for the FEM vs. 24.6% for the HRS). Household wealth differs in 2012 ($362,000 for the FEM vs $339,000 for the HRS).  Earnings and capital income 
do not differ significantly between the two populations.

On the whole, the cross-validation exercise is reassuring.  Comparing simulated outcomes to actual outcomes using a set of transition models estimated on a 
separate population reveals that the majority of outcomes of interest are not statistically different.  In cases where they are, the practical difference is 
potentially low.
