\subsection{Obesity reduction scenario}
In addition the to the status quo scenario, the Future Elderly Model can be 
used to estimate the effects of numerous possible policy changes. One such set 
of policy simulations involves changing the trends of risk factors for chronic 
conditions. This is implemented by altering the incoming cohorts. A useful 
example is an obesity reduction scenario which rolls back the prevalence of 
obesity among 50 year-olds to its 1978 level by 2030, where it remains until 
the end of the scenario, in 2050. This is accomplished by reversing the annual
rates of change for BMI category, hypertension, and diabetes shown in Table 
\ref{tab:prevalence_1978_2004}.  As seen in Table \ref{tab:obesity_results}, 
this will change the prevalence of obesity among the age 50+ in 2050. As 
compared with the status quo estimates (Table \ref{tab:status_quo_results}) the 
FEM predicts that by 2050, this will result in a change in the amount of Social 
Security benefits as well as changing combined Medicare and Medicaid expenditures.

