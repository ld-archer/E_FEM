\subsection{Drug Expenditures}
\label{sec:model_development_rxexp}

\subsubsection{Drug Expenditures - MEPS}
\label{sec:model_development_rxexp_meps}
AHRQ produces a file of consolidated annual expenditures for each Medical Expenditure Panel Survey respondent in each calendar year.  
The total drug expenditure variable sums all amounts paid out-ot-pocket and by third party payers for each prescription
purchased in that year.  For comparison across years, we convert all amounts to 2010 dollars using the Medical CPI.

\subsubsection{Drug Expenditures - MCBS}
\label{sec:model_development_rxexp_mcbs}
The Medicare Current Beneficiary Survey produces a Prescribed Medicine Events file at the individual-event level, with cost
and utilization of prescribed medicines for the MCBS community population.  Collapsing this file to the individual provides an
estimate of prescription drug cost for each person.  For comparison across years, we convert all amounts to 2010 dollars using the Medical CPI.


There are two caveats to working with these data.  The first caveat regards how to handle the "ghost" respondents.  
"Ghosts" are individuals who enroll in Medicare, but were not
asked cost and use questions in the year of their enrollment.  For some outcomes, such as medical expenditures, the MCBS makes an 
effort to impute.  For others, such as drug utilization and expenditures, the MCBS does not.  We imputed annual drug expenditures 
for the ghosts, but for certain age ranges the drug expenditures were not reasonable.  This had the biggest effect on the 65 and 66 
year olds, for two reasons.  The first is that the 65 and 66 year olds are more likely to be ghosts, as 65 is the typical age of 
enrollment for Medicare.  The second is that the 65 and 66 year olds used for imputation (i.e., the non-"ghosts") are different.  To 
be fully present in MCBS at age 65 would require enrolling in Medicare before age 65, which happen through a different channel, such
as qualifying for Medicare due to receiving disability benefits from the federal government.  

The second caveat relates to the filling in zeroes for individuals with no utilization data, but who were enrolled.  We assumed that
individuals who were not ghosts and who did not appear on the Prescribed Medicine Events file had zero prescription expenditures.  
  

\subsubsection{Drug Expenditures - Estimation}
\label{sec:model_development_rxexp_estimation}
Due to the complexities of the age 65-66 population in the MCBS, we chose to estimate the drug expenditure models using the MEPS for 
individuals 51 to 66 and the MCBS for individuals 67 and older.  Individuals under age 65 receiving Medicare due to disability are 
estimated separately.  Since there are a number of individuals with zero expenditures, we estimate the models in two stages.  The 
first stage is a probit predicting any drug expenditures and the second is an ordinary least squares model predicting the amount, 
conditional on any.  Coefficient estimates and marginal effects are shown in the accompanying Excel workbook.

