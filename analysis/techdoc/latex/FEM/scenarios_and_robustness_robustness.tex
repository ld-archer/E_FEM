\subsection{Robustness}
\label{sec:robustness}
A number of the restrictions on health transitions prove to be statistically significant in the health transitions model. 
%(see the table of estimates \url{https://healthpolicy.box.com/shared/static/hb95i1e0i21kiz4lvd87278i5s2zpdhc.xml})
Thus, we test the robustness of the FEM by analyzing the policy simulations with and without 
some of the restrictions. We test the robustness in two steps. First, we look at the effect of removing the restrictions on health outcomes 
affecting other health transitions, and compare the simulation results both for the status quo model, and for the obesity scenario discussed 
above. The model without health restrictions has less than 1\% difference in predictions relating to government expenditures in both 2030 
and 2050 (as shown in Table 28). There are slightly higher, yet still quite small increases in simulated government revenues.  Importantly, Table 29, which shows the effect 
of the obesity scenario without the health restrictions, finds very similar effects on social security and Medicare/Medicaid as under the 
FEM (2.35\% as compared with 2.28\% increase in social security benefits and a 4.31\% as compared with 4.37\% decrease in combined 
Medicare/Medicaid expenditures). 

We similarly test the effect of the economic restrictions on health transitions. The economic restrictions have a larger impact on 
projected expenditures and revenues than did the health restrictions (see Table 30). However, they still have little effect on the 
conclusions drawn from the obesity scenario. The obesity scenario, in Table 31, where we remove the economic restrictions, implies the same 
2.28\% increase in social security expenditures, and a very similar 4.39\% drop in Medicaid/Medicare expenditures in 2050. Thus, while the 
imposed restrictions have some, often small, impact on the FEM baseline estimates, they have little effect on the conclusions from the 
obesity reduction scenario.

