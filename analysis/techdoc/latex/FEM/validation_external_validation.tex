\subsection{External Validation}
%NOTE: THIS SECTION NEEDS TO BE VALIDATED

The external validation exercise compares FEM full population simulations beginning in 2004 to external sources.  Here we focus on per capita benefits received from 
Social Security, Disability, and Supplemental Security Income, followed by Medicare and Medicaid.

\subsubsection{Benefits from Social Security Administration}

Conditional on a simulant receiving benefits, the FEM algorithmically assigns benefits for Old Age and Survivors, Supplemental Security Income, and Disability. Here, 
we compare simulation results to SSA figures.

For Old Age and Survivor benefits, we compare to the Social Security Administration�s December 2012 Monthly Statistical Snapshot.  Table 2 of that document indicates 
that the average OASI monthly benefit was \$1194.  FEM forecasts \$1182 for the average beneficiary for 2012.  

For Supplemental Security Income we compare to Table 3 of the December 2012 Monthly Statistical Snapshot, focusing on the 65 and older population, as that is the 
only category that is directly comparable.  SSA reports that the average monthly benefit for December of 2012 was \$417.  FEM assigns \$415 to those receiving SSI.

SSA does not report a disability figure that is directly comparable to FEM forecasts.  However, SSA reports average Disability benefits by age, as well as the 
number of individuals receiving benefits at each age.  This allows us to construct the average benefit for workers 51 and older.  Based on this calculation, the 
average disabled worker 51 and older received a benefit of \$1212 in December of 2012.  Spouses of disabled workers can also receive a benefit (SSA reports a 
benefit of \$304 for spouses of disabled workers for all ages).  The 2012 FEM forecast for the average DI beneficiary, which includes both workers and 
their spouses, is \$1102.  

\subsubsection{Benefits from Medicare and Medicaid}

For medical spending, we compare FEM forecasts in 2010 to National Health Expenditure Accounts measures from 2010, the most recent year for which these data 
are available.  NHEA reports total amounts by age range, which we then convert to per capita measures using the 2010 Census.  We focus on the 65-84 and 85 plus 
populations, as they are directly comparable to FEM forecasts.  We also aggregate the two groups to produce a 65 plus average.  FEM is similar to NHEA for the 
65 plus population for Medicare (\$10473 for NHEA, \$10494 for FEM) and total medical spending (\$19265 for NHEA, \$19056 for FEM).  FEM estimates are higher for 
Medicaid spending (\$2141 for NHEA, \$2818 for FEM).

