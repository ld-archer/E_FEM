\subsection{Data for trends in entering cohorts}
\label{sec:data_sources_trends_and_baseline_entering_cohorts}
We used a multitude of data sources to compute U.S. trends. First, we used NHIS for chronic conditions 
and applied the methodology discussed in \citep{goldman2004health}. The method consists of projecting the 
experience of younger cohorts into the future until they reach age 51. The projection method is 
tailored to the synthetic cohorts observed in NHIS. For example, we observe a representative sample of 
age 35 individuals born in 1945 in 1980. We follow their disease patterns in 1980 to 1981 surveys by 
then selecting those aged 36 in 1981, accounting for mortality, etc.  

We then collected information on other trends, i.e. for obesity and smoking, from 
other studies \citep{honeycutt2003dynamic,levy2006smoking,poterba2009decline,ruhm2007current,mainous2007impact}. 
Table \ref{tab:cohort_projection_data_methods} presents the sources and Table \ref{tab:baseline_trends} presents the trends we use in the baseline scenario. 
Table \ref{tab:prevalence_1978_2004} presents the prevalence of obesity, hypertension, diabetes, and current smokers in 1978 and 
2004, and the annual rates of change from 1978 to 2004.  We refer the readers to the analysis in 
\citet{goldman2004health} for information on how the trends were constructed.

