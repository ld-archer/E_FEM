\subsection{Medical costs estimation}
\label{sec:govt_revenue_and_expenditures_medcost_estimation}
In the FEM, a cost module links a person's current state--demographics, economic status, current health, risk factors, and functional status 
to 4 types of individual medical spending. The FEM models: total medical spending (medical spending from all payment sources), Medicare 
spending\footnote{We estimate annual medical spending paid by specific parts of 
Medicare (Parts A, B, and D) and sum to get the total Medicare expenditures.}, 
Medicaid spending (medical spending paid by Medicaid), and out of pocket spending 
(medical spending by the respondent). These estimates are based on pooled weighted 
least squares regressions of each type of spending on risk factors, self-reported 
conditions, and functional status, with spending inflated to constant dollars 
using the medical component of the consumer price index.  We use the 2000-2010 
Medical Expenditure Panel Survey 
% \todo : Change the text to be something like "roughly X respondents per panel, with a new panel starting entering every Y months and lasting Z months between YYYY and YYYY." That way, it is always correct and all we do is change the years of data we're looking at. 
for these regressions for persons 
not Medicare eligible, and the 2000-2010 Medicare Current Beneficiary Survey 
% \todo : Change the text to be something like "roughly X respondents per panel, with a new panel starting entering every Y months and lasting Z months between YYYY and YYYY." That way, it is always correct and all we do is change the years of data we're looking at. 
for spending for those that are eligible for Medicare. Those eligible for 
Medicare include people eligible due to age (65+) or due to disability status. Comparisons of prevalences and question wording across these different sources are provided in Tables \ref{tab:svy_disease_prevalence} and \ref{tab:svy_disease_questions}, respectively.

In the baseline scenario, this spending estimate can be interpreted as the resources consumed by the individual given the manner in which 
medicine is practiced in the United States during the post-part D era (2006-2010). 
Models are estimated for total, Medicaid, out of pocket spending, and for the Medicare spending. 
These estimates only use the MCBS dataset.

Since Medicare spending has numerous components (Parts A and B are considered here), models are needed to predict enrollment. In 2004, 
98.4\% of all Medicare enrollees, and 99\%+ of aged enrollees, were in Medicare Part A, and thus we assume that all persons eligible for 
Medicare take Part A. We use the 2007-2010 MCBS to model take up of Medicare Part B for both new enrollees into Medicare, as well as 
current enrollees without Part B. Estimates are based on weighted probit regression on various risk factors, demographic, and economic 
conditions. The HRS starting population for the FEM does not contain information on Medicare enrollment. Therefore another model of Part B 
enrollment for all persons eligible for Medicare is estimated via a probit, and used in the first year of simulation to assign initial Part 
B enrollment status. Estimation results are shown in estimates table. The MCBS data over represents the portion enrolled in Part B, having a 97\% 
enrollment rate in 2004 instead of the 93.5\% rate given by Medicare Trustee's Report. In addition to this baseline enrollment probit, we 
apply an elasticity to premiums of -0.10, based on the literature and simulation 
calibration for actual uptake through 2009 \citep{atherly2004effect,buchmueller2006price}. 
The premiums are computed using average Part B costs from the previous time step 
and the means-testing thresholds established by the ACA.

Since both the MEPS and MCBS are known to under-predict medical spending (see, e.g., \citeauthor{selden2008aligning}, \citeyear{selden2008aligning}, and references therein), we applied adjustment factors to the predicted three types of 
individual medical spending so that the predicted per-capita spending in FEM equal the corresponding spending in National 
Health Expenditure Accounts (NHEA) for age group 55-64 in year 2004 and ages 65 and over in year 2010, respectively. Table 
\ref{tab:NHEA_adjustment} shows how these adjustment factors were determined by using the ratio of expenditures in the NHEA to 
expenditures predicted in the FEM.  

Since 2006, the Medicare Current Beneficiaries Survey (MCBS) contains data on Medicare Part D. The data gives the capitated Part D payment and 
enrollment. When compared to the summary data presented in the CMS 2007 Trustee Report, the 2006 per capita cost is comparable between the MCBS 
and the CMS. However, the enrollment is underestimated in the MCBS, 53\% compared to 64.6\% according to CMS. 

% \todo : add the tables or plots of Part D enrollment, payment, etc.

A cross-sectional probit model is estimated using years 2007-2010 to link demographics, economic status, current health, and 
functional status to Part D enrollment - see the estimates table. To account for both the initial under reporting of Part D 
enrollment in the MCBS, as well as the CMS prediction that Part D enrollment will rise to 75\% by 2012, the constant in the probit model is 
increased by 0.22 in 2006, to 0.56 in 2012 and beyond.  The per capita Part D cost in the MCBS matches well with the cost reported from 
CMS. An OLS regression using demographic, current health, and functional status is estimated for Part D costs based on capitated payment amounts.

The Part D enrollment and cost models are implemented in the Medical Cost module. The Part D enrollment model is executed conditional on 
the person being eligible for Medicare, and the cost model is executed conditional on the enrollment model leading a true result, after the 
Monte Carlo decision. Otherwise the person has zero Part D cost. The estimated Part D costs are added with Part A and B costs to obtain 
total Medicare cost, and any medical cost growth assumptions are then applied.

