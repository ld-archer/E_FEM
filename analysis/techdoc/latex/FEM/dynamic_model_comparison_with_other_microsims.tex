\subsection{Comparison with other prominent microsimulation models of health expenditures}
The FEM is unique among existing models that make health expenditure projections. It is the only model 
that projects health trends rather than health expenditures. It is also the only model that generates 
mortality out of assumptions on health trends rather than historical time series.

\subsubsection{CBOLT Model}
The Congressional Budget Office (CBO) uses time-series techniques to project health expenditure 
growth in the short term and then makes an assumption on long-term growth. They use a long term 
growth of excess costs of 2.3 percentage points starting in 2020 for Medicare. They then assume a 
reduction in excess cost growth in Medicare of 1.5\% through 2083, leaving a rate of 0.9\% in 2083. 
For non-Medicare spending they assume an annual decline of 4.5\%, leading to an excess growth rate in 
2083 of 0.1\%. 

\subsubsection{Centers for Medicare and Medicaid Services}
The Centers for Medicare and Medicaid Services (CMS) performs an extrapolation of medical expenditures 
over the first ten years, then computes a general equilibrium model for years 25 through 75 and 
linearly interpolates to identify medical expenditures in years 11 through 24 of their estimation. 
The core assumption they use is that excess growth of health expenditures will be one percentage point 
higher per year for years 25-75 (that is if nominal GDP growth is 4\%, health care expenditure growth 
will be 5\%).

