\subsection{Taxes}
We consider Federal, State and City taxes paid at the household level. We also calculate Social Security taxes and Medicare taxes. HRS 
respondents are linked to their spouse in the HRS simulation. We take program rules from the OECD's Taxing Wages Publication for 2004. 
Households have basic and personal deductions based on marital status and age ($>$65). Couples are assumed to file jointly. Social Security 
benefits are partially taxed. The amount taxable increases with other income from 50\% to 85\%. Low income elderly have access to a special 
tax credit and the earned income tax credit is applied for individuals younger than 65. We calculate state and city taxes for someone 
living in Detroit, Michigan. The OECD chose this location because it is generally representative of average state and city taxes paid in 
the U.S. Since Social Security administrative data cannot be used jointly with Geocoded information in the HRS, we apply these hypothetical 
taxes to all respondents.

At the state level, there is a basic deduction for each member of the household (\$3,100) and taxable income is taxed at a flat rate of 4\%. 
At the city level, there is a small deduction of \$750 per household member and the remainder is taxed at a rate of 2.55\%. There is however 
a tax credit that decreases with income (20\% on the first 100\$ of taxes paid, 10\% on the following 50\$ and 5\% on the remaining portion). 

We calculate taxes paid by the employee for Old-Age Social Insurance (SS benefits and DI) and Medicare (Medicaid and Medicare). It does not 
include the equivalent portion paid by the employer. OASI taxes of 6.2\% are levied on earnings up to \$97,500 (2004 cap) while the Medicare 
tax (1.45\%) is applied to all earnings.

% \todo : add payroll tax description here