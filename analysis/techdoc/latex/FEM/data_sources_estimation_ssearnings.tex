\subsection{Social Security covered earnings files}
To get information on Social Security entitlements of respondents, we match the HRS data to the 
Social Security Covered Earnings files of 1992, 1993, 1998, 2004 and 2006 which provides information on 
earning histories of respondents as well as their entitlement to future Social Security benefits. We 
then construct the average indexed monthly earnings (AIME), the basis for the determination of benefit 
levels, from these earning histories. The AIME is constructed by first indexing using the National 
Wage Index (NWI) to the wage level when the respondent turns age 60. If this occurs after 2008, we 
project the evolution of the NWI using the average annual rate of change of the last 20 years (2.9\% 
nominal). We then take the 35 highest years (if less than 35 years are available, remaining years are 
considered zero earning years) and take the average. We then convert back this annual amount on a 
monthly basis and convert back to \$2004 U.S. dollars using the CPI. Quarters of coverage, which 
determine eligibility to Social Security, are defined as the sum of posted quarters to the file. A 
worker is eligible for Social Security if he has accumulated at least 40 quarters of coverage. A 
worker roughly accumulates a quarter of coverage for every \$4000 of coverage earnings up to a maximum 
of 4 per year. Not all respondents agree to have their record matched. Hence, there is the potential 
for non-representativeness. However, recent studies show that the extent of non-representativeness is 
quite small and that appropriate weighting using HRS weights mostly corrects for this problem 
\citep{kapteyn2006effects}.