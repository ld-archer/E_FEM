\subsection{Social Security benefits}
Workers with 40 quarters of coverage and of age 62 are eligible to receive their retirement benefit. The benefit is calculated based on the 
Average Indexed Monthly Earnings (AIME) and the age at which benefits are first received. If an individual claims at his normal retirement 
age (NRA) (65 for those born prior to 1943, 66 for those between 1943 and 1957, and 67 thereafter), he receives his Primary Insurance 
Amount (PIA) as a monthly benefit. The PIA is a piece-wise linear function of the AIME. If a worker claims prior to his NRA, his benefit is 
lower than his PIA. If he retires after the NRA, his benefit is higher. While receiving benefits, earnings are taxed above a certain 
earning disregard level prior to the NRA. An individual is eligible to half of his spouse�s PIA, properly adjusted for the claiming age, 
if that is higher than his/her own retirement benefit. A surviving spouse is eligible to the deceased spouse�s PIA. Since we assume prices 
are constant in our simulations, we do not adjust benefits for the COLA (Cost of Living Adjustment) which usually follows inflation. We 
however adjust the PIA bend points for increases in real wages. 
