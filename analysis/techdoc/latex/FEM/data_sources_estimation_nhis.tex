\subsection{National Health Interview Survey}
The National Health Interview Survey (NHIS) contains individual-level data on height, weight, smoking 
status, self-reported chronic conditions, income, education, and demographic variables. It is a 
repeated cross-section done every year for several decades. But the survey design has been 
significantly modified several times. Before year 1997, different subgroups of individuals were asked 
about different sets of chronic conditions, after year 1997, a selected sub-sample of the adults were 
asked a complete set of chronic conditions. The survey questions are quite similar to that in HRS. As 
a result, for projecting the trends of chronic conditions for future 51/52 year-olds, we only use data 
from 1997 to 2010. A review of survey questions is provided in Table \ref{tab:svy_disease_questions}. Information on 
weight and height were asked every year, while information on smoking was asked in selected years before 
year 1997, and has been asked annually since year 1997. 

FEM uses NHIS to project prevalence of chronic conditions in future cohorts of 51-52 year olds.  The
method is discussed in Sections \ref{sec:data_sources_trends_and_baseline_entering_cohorts} and \ref{sec:new_cohorts_model_empirical_strategy}.  FEM
also relies on the Medical Expenditure Panel Survey, a subsample of NHIS respondents, for model estimation.
See section \ref{sec:data_sources_estimation_meps} for a description.

