\documentclass{article}
\usepackage{geometry}
\usepackage{fancyhdr}
\usepackage{amsmath,amsthm,amssymb}
\usepackage{graphicx}
\usepackage{hyperref}
\usepackage{lipsum}



\title{The Future America Model: Technical Documentation }
\author{Bryan Tysinger
\and
Duncan Leaf}
\date{\today}

\begin{document}
\maketitle
\tableofcontents
\listoffigures
\listoftables

\newpage

\section{Functioning of the Dynamic Model}
\subsection{Background}
History of the FEM, funding sources.

\subsection{Overview}
The defining characteristic of the model is the modeling of real rather than synthetic cohorts, all of whom 
are followed at the individual level. This allows for more heterogeneity in behavior than would be allowed 
by a cell-based approach. Also, since the PSID interviews both respondent and spouse, we can link records to 
calculate household-level outcomes, which depend on the outcomes of both spouses. 

The model has three core components: 

\begin{itemize}
\item The initial cohort module predicts the economic and health outcomes of new cohorts of 25/26 year-olds. This 
module takes in data from the Panel Survey of Income Dynamics (PSID) and trends calculated from other sources. It 
allows us to �generate� cohorts as the simulation proceeds, so that we can measure outcomes for the age 25+ 
population in any given year. 
\item The transition module calculates the probabilities of transiting across various health states and financial 
outcomes. The module takes as inputs risk factors such as smoking, weight, age and education, along with lagged 
health and financial states. This allows for a great deal of heterogeneity and fairly general feedback effects. 
The transition probabilities are estimated from the longitudinal data in the PSID. 
\item The policy outcomes module aggregates projections of individual-level outcomes into policy outcomes such as 
taxes, medical care costs, and disability benefits. This component takes account of public 
and private program rules to the extent allowed by the available outcomes.  
\end{itemize}

Figure 1 provides a schematic overview of the model. We start in 2009 with an initial population aged 25+ taken 
from the PSID. We then predict outcomes using our estimated transition probabilities (see section \ref{transition}). Those who 
survive make it to the end of that year, at which point we calculate policy outcomes for the year. We then move 
to the following time period (two years later), when a new cohort of 25 and 26 year-olds enters (see section \ref{newcohorts}). 
This entrance forms the new age 25+ population, which then proceeds through the transition model as before. This 
process is repeated until we reach the final year of the simulation. 

\subsection{Comparison with other prominent microsimulation models}
The precursor to the FAM, the FEM, was unique among models that make health expenditure projections.  It was the only 
model that projected health trends rather than health expenditures.  It was also unique in generating mortality projections
based on assumptions about health trends rather than historical time series.

FAM extends FEM to younger ages, adding additional dimensions to the simulation.  Events over the life course, such as marital status
and childbearing are simulated.  Labor force participation is modeled in greater detail, as is education.

\section{Data Sources for Estimation}
The Panel Survey of Income Dynamics is the main data source for the model.
\subsection{Panel Survey of Income Dynamics}
The 1999-2011 waves of the Panel Survey of Income Dynamics (PSID) are used to estimate the transition models.  PSID interviews
occur every two years.  We use all individuals twenty-five or older who have formed their own households in the analysis, all heads, wives, and
"wives" in PSID terms.  When appropriately weighted, this PSID is representative of the U.S. population 25+ and older.

\subsection{Medical Expenditure Panel Survey}
The MEPS, beginning in 1996, is a set of large-scale surveys of families and individuals, their medical providers (doctors, 
hospitals, pharmacies, etc.), and employers across the United States. The Household Component (HC) of the MEPS provides 
data from individual households and their members, which is supplemented by data from their medical providers. The Household 
Component collects data from a representative sub sample of households drawn from the previous year's National Health Interview 
Survey (NHIS). Since NHIS does not include the institutionalized population, neither does MEPS: this implies that we can only 
use the MEPS to estimate medical costs for the non-elderly population. Information collected during household interviews 
include: demographic characteristics, health conditions, health status, use of medical services, sources of medical payments, 
and body weight and height. Each year the household survey includes approximately 12,000 households or 34,000 individuals. 
Sample size for those aged 25-64 is about ???.  MEPS has comparable measures of social-economic (SES) variables as those in 
PSID, including age, race/ethnicity, educational level, census region, and marital status. 

\subsection{Medicare Current Beneficiary Survey}
The MCBS is a nationally representative sample of aged, disabled and institutionalized Medicare beneficiaries.  The MCBS 
attempts to interview each respondent twelve times over three years, regardless of whether he or she resides in the community, 
a facility, or transitions between community and facility settings. The disabled (under 65 years of age) and oldest-old (85 
years of age or older) are over-sampled. The first round of interviewing was conducted in 1991. Originally, the survey was a 
longitudinal sample with periodic supplements and indefinite periods of participation. In 1994, the MCBS switched to a rotating 
panel design with limited periods of participation. Each fall a new panel is introduced, with a target sample size of 12,000 
respondents and each summer a panel is retired. Institutionalized respondents are interviewed by proxy.  The MCBS contains 
comprehensive self-reported information on the health status, health care use and expenditures, health insurance coverage, 
and socioeconomic and demographic characteristics of the entire spectrum of Medicare beneficiaries.  Medicare claims data for 
beneficiaries enrolled in fee-for-service plans are also used to provide more accurate information on health care use and 
expenditures.  

We estimate cost and utilization in the "part D era," using data from 2006 to 2010.   

\subsection{Health and Retirement Study}
The Health and Retirement Study (HRS), waves 1998-2010 are pooled with the PSID population to estimate the mortality models. Interviews 
in the HRS occur every two years.  We use the dataset created by RAND (RAND HRS, version M) for the analysis. When appropriately weighted, 
the HRS is representative of U.S. households where at least one member is at least 51. 

\section{Data Sources for Trends and Baseline Scenario}
\subsection{Data for Trends in Replenishing Cohorts}
\subsubsection{American Community Survey}
\subsubsection{National Health and Nutrition Examination Survey}

\subsection{Data for Other Projections}
\subsubsection{Wage growth, medical cost growth}
\subsubsection{Mortality reduction}

\subsection{Demographic Adjustments}
\subsubsection{Census for adjusting new cohort weights and for simulating immigration}

\section{Estimation}
\subsection{Transition Models}
\label{transition}

\section{Model for New Cohorts}
\label{newcohorts}
\subsection{Information available and empirical strategy}
\subsection{Model and Estimation}

\section{Government Revenue and Expenditures}
\subsection{Medical Costs Estimation}
\subsection{Approach to SS, DI, SSI}
\subsection{Taxes}

\section{Implementation of the Future America Model}
\subsection{Description of the Simulation}
\subsection{Assumptions}
\subsubsection{Wage Growth}
\subsubsection{Medical Cost Growth}
\subsubsection{Mortality Reduction}
\subsection{Adjustments to calibrate health costs (NHEA)}

\section{Internal Validation}
\subsection{Cross-validation}

\section{External Validation}
\subsection{Population Projections}
\subsection{Health Forecasts}
\subsection{Medical Costs and Participation}
\subsubsection{Medicare A/B/D and Medicaid enrollment and spending}
\subsubsection{Private health insurance enrollment}
\subsection{Fertility and Marriage Rates}




\end{document}

