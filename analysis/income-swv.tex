\documentclass{article}
\usepackage{amsfonts}
\usepackage{savetrees}
\usepackage{hyperref}
\usepackage[section]{placeins}
\usepackage{verbatim}
\usepackage{color}

\title{Debugging FEM Economic Data}

\begin{document}

\begin{Rcode}{!echo,hide}
library(foreign)
library(ggplot2)
stat_sum_single <- function(fun, geom='point', ...) { 
  stat_summary(fun.y=fun, colour='red', geom=geom, size = 3, ...) 
} 
agegrp <- function(ages, levels=list(b60=0:60,
f60to65=60:65, f65to70=65:70, f70to75=70:75,
f75to80=75:80, f80to85=80:85, o85=85:200)) {
 outvec <- 'none'
 for(i in 1:length(levels)) {
    outvec[min(levels[[i]]) <= ages & ages < max(levels[[i]])] <- names(levels)[i]
  }
  return(outvec)
}
\end{Rcode}

\maketitle
\listoftables
\listoffigures

\section{Variables of Interest}
Table~\ref{tab:variables} shows the variables of interest along with quick descriptions. In addition to the variables presented, there are initial value variables (prefaced with an 'f'), lag value variables ('l'), and log-transformed variables ('log').
\begin{table}[ht]
\centering
\caption{Economic Variables in the FEM}
\begin{tabular}{|c|c|l|}\hline
\textbf{Name} & \textbf{Range} & \textbf{Description} \\\hline
anydb & $\{0, 1\}$ & any defined benefit from current job\\\hline
anydc & $\{0,1\}$ & any defined contribution from current job\\\hline
db\_tenure & $[-2, 54.5]$ & defined benefit tenure\\\hline
dbage & $\mathbb{R}^+$ & age started claiming pension\\\hline
dbclaim & $\{0,1\}$ & currently claiming defined benefit pension\\\hline
dbpen & $\mathbb{R}^+$ & defined benefit pension amount\\\hline
dcwlth & $\mathbb{R}^+$ & defined contribution wealth in thousands\\\hline
dcwlthx & $\mathbb{R}^+$ & defined contribution wealth in thousands\\\hline
diben & $\mathbb{R}^+$ & amount of disability insurance benefits\\\hline
diclaim & $\{0, 1\}$& currently claiming SSDI\\\hline
gross & $\mathbb{R}^+$ & gross income\\\hline
hatota & $\mathbb{R}$ & household wealth in thousands\\\hline
hatotax & $[0, 2000]$ & household wealth in thousands, capped\\\hline
agi & $\mathbb{R}^+$ & estimated adjusted gross income\\\hline
hhearn & $\mathbb{R}^+$ & total household earnings\\\hline
hhttlinc & $\mathbb{R}^+$ & total household income, currently set equal to hhearn\\\hline
hicap & $\mathbb{R}$ & estimated household capital income\\\hline
hicap\_nonzero & $\{0, 1\}$ & hicap nonzero\\\hline
iearn & $\mathbb{R}^+$ & individual earnings in thouseands\\\hline
iearnx & $[0, 2000]$ & individual earnings in thousands, capped at 2000\\\hline
isret & $\mathbb{R}^+$ & R SoCSec Retirement Income\\\hline
net & $\mathbb{R}$ & net income\\\hline
ry\_earn & $[0, 2000000]$ & iearnx * 1000\\\hline
rssclyr & $\mathbb{R}^+$ & year person first claimed Social Security\\\hline
ssben & $\mathbb{R}^+$ & SS benefits\\\hline
ssclaim & $\{0, 1\}$ & currently claiming OASI\\\hline
ssiben & $\mathbb{R}^+$ & SSI benefits\\\hline
ssiclaim & $\mathbb{R}^+$ & currently claiming SSI\\\hline
wlth\_nonzero & $\{0, 1\}$ & hatota nonzero\\\hline
work & $\{0, 1\}$ & person currently working\\\hline
\end{tabular}
\label{tab:variables}
\end{table}

\section{Model Algorithm}

\subsection{New Cohorts}
\label{sec:new}
\begin{itemize}
\item all incoming cohorts have the same basic distribution of earnings, household wealth, and defined contribution plan wealth
\item earnings data are inflated according to the NWI factor, which is the real growth in wages from the reference year to the year when the cohort enters the simulation
\item household wealth data are inflated by CPI to the year when the cohort enters the simulation
\item DC wealth is not modified
\item household capital income is estimated via a linear regression
\end{itemize}
\subsubsection{Income Initialization}
\label{sec:init}
\begin{eqnarray}
\rm fiearn &=& {\rm iearnx} * {\rm work}\\
\rm fiearnx &=& \max\left(\min\left(200, {\rm fiearn}\right), 0\right)\\
\rm flogiearn &=& {\rm arcsinh}\left(\rm fiearn\right)/100\\
\rm flogiearnx &=& {\rm arcsinh}\left(\max\left(\min\left(200, {\rm fiearn}\right), 0\right)\right)\\
\rm iearn &=& {\rm iearnx} * \Delta_{\rm NWI} * {\rm work}\\
\rm iearnx &=& \max(\min(200, {\rm iearn}), 0)\\
\rm logiearn &=& {\rm arcsinh}\left(\rm iearn\right)/100\\
\rm logiearnx &=& {\rm arcsinh}\left(\rm iearnx\right)/100\\
\rm gross &=& {\rm iearnx} * 1000 + {\rm iearnx}_{\rm spouse} * 1000
\end{eqnarray}
In addition, hicap\_nonzero and hicap are predicted by regression
models.

\subsubsection{Household Wealth Initialization}
\begin{eqnarray}
\rm hatotax &=& \min\left(2000, {\rm hatota}\right)\\
\rm loghatota &=& {\rm arcsinh}\left(\rm hatota\right)/100\\
\rm loghatatax &=& {\rm arcsinh}\left(\rm hatotax\right)/100
\end{eqnarray}

\subsubsection{Benefits Claiming Initialization}
\begin{eqnarray}
\rm rssclyr &=& \left\{ \begin{tabular}{ll}2100&{\rm
      missing(rssclyr)}\\rssclyr&{\rm else}\end{tabular}\right.\\
\rm ssclaim &=& {\rm rssclyr} \leq {\rm year}\\
\rm rdbclyr &=& \left\{ \begin{tabular}{ll}2100&{\rm
      missing(rdbclyr)}\\rdbclyr&{\rm else}\end{tabular}\right.\\
\rm dbclaim &=& {\rm rdbclyr} \leq {\rm year}\\
\rm ldiclaim &=& \rm diclaim
\end{eqnarray}

\subsection{Single Timestep Flow}
Here are the different modules, in the order in which they are run,
and the relevant encomic calculations done in each.

\subsubsection{Health Module}
The following regressions are run conditionally:
\begin{tabular}{ll}
dbclaim &if fanydb AND NOT ldbclaim\\
ssclaim &if lage $\leq$ 68 AND NOT lssclaim AND
    lage $\geq$ 60\\
diclaim & if lage $<$ 63\\
work &if lage $<$ 73\\
\end{tabular}

\subsubsection{Interventions Module}
This module unconditionally runs the following regressions:
\begin{tabular}{l}
work\\
wlth\_nonzero\\
hicap\_nonzero\\
\end{tabular}

\subsubsection{Earnings Module}
The following regressions are run and the derivative variables
calculated as in \ref{sec:init}:
\begin{tabular}{ll}
hatota & if wlth\_nonzero\\
hicap & if hicap\_nonzero\\
iearn & if work\\
\end{tabular}

\subsubsection{Immigration Module}
This module adjusts the weights of all the records according to the
trends given in the scenario file.

\subsubsection{Initialization Module}
Here is where the calculations from Section~\ref{sec:new} actually
take place.

\subsubsection{Crossectional Module}
In the alzheimers branch, this is where the quality adjusted life
years cross-sectional model is run.

\subsubsection{Economic Outcomes Module}
The following are calculated for everyone:
\begin{eqnarray}
{\rm rssclyr} &=& \left\{\begin{tabular}{ll}year&if ssclaim\\2100&else\end{tabular}\right.\\
{\rm ry\_earn} &=& {\rm iearnx} * 1000\\
{\rm hhwealth} &=& {\rm hatotax} * 1000\\
{\rm db\_tenure} &=& {\rm db\_tenure} + \Delta_T ~{\rm if work AND
  fanydb}\\
{\rm dbpen} &=& {\rm DbBenefit}()\\
{\rm hhearn} &=& {\rm ry\_earn} + {\rm ry\_earn}_{\rm spouse}\\
{\rm hhttlinc} &=& {\rm hhearn}
\end{eqnarray}

\subsubsection{Government Expenses Module}
Government expenditures are generally benefits received by people:
\begin{eqnarray}
{\rm ssben} &=& \left\{\begin{tabular}{ll}12 * {\rm
      SSCalculator.SSBenefit}&{\rm if ssclaim}\\0&{\rm
      else}\end{tabular}\right.\\
{\rm diben} &=& \left\{\begin{tabular}{ll}{\rm
      GovExpModule.DiBenefit}&{\rm if diclaim}\\0&{\rm
      else}\end{tabular}\right.\\
{\rm ssiclaim} &=& \textbf{\color{red}POSSIBLE PROBLEM: CHECKING WITH IGOR}\\
{\rm ssiben} &=& \left\{\begin{tabular}{ll}0&{\rm NOT
      ssiclaim}\\450*12&{\rm age} < 65\\350*12&{\rm age} >=
    65\end{tabular}\right.\\
{\rm gross}&=&\\
{\rm agi}&=&\\
{\rm net}&=&
\end{eqnarray}
\textbf{\color{red} Should SSIBEN be calculated based on age or medicare
  eligibility? Should it change via an inflating factor?}

\subsubsection{Medical Costs Module}
No economic variables are calculated here.

\section{The Poverty Problem}

\begin{Rcode}{!echo,savefig,label=exemptions}
base <- read.dta('../output/ACA/summary.dta')
base <- subset(base, year >= 2010)
a <- base[,'ttl_mcare_pta_exempt'] / base[,'pop_medicare']
b <- base[,'ttl_mcare_ptb_exempt'] / base[,'pop_medicare']
dat <- data.frame(pta_exempt=a, ptb_exempt=b,
ttl_ptb=base[,'ttl_mcare_ptb_exempt'], year=base[,'year'],
pop=base[,'pop_medicare'])
ggplot(dat, aes(x=year)) + 
geom_line(aes(y=ptb_exempt, colour='Fraction')) + 
geom_line(aes(y=ttl_ptb, colour='Total Exempt (M)')) +
geom_line(aes(y=pop, colour='Medicare Eligible (M)')) +
facet_grid(colour ~ ., scales='free_y') +
opts(legend.position='bottom') + labs(colour='Series', y='')
\end{Rcode}

\begin{Rcode}{!echo,savefig,label=wlthnonzero}
agesvec <- c('5059','6064','6569','7074','7579','80')
wlthnames <- paste('fraction_wlth_nonzero_', agesvec, sep='')
wlthnonzero = base[,c('year',wlthnames)]
ggplot(wlthnonzero, aes(x=year, ymin=0, ymax=1)) +
geom_line(aes(y=fraction_wlth_nonzero_5059, colour='50-59')) + 
geom_line(aes(y=fraction_wlth_nonzero_6064, colour='60-64')) + 
geom_line(aes(y=fraction_wlth_nonzero_6569, colour='65-69')) +
geom_line(aes(y=fraction_wlth_nonzero_7074, colour='70-74')) +
geom_line(aes(y=fraction_wlth_nonzero_7579, colour='75-79')) +
geom_line(aes(y=fraction_wlth_nonzero_80, colour='80+')) +
opts(legend.position='bottom') +
labs(y='Fraction with Nonzero Wealth', colour='Age Group')
\end{Rcode}

\begin{Rcode}{!echo,savefig,label=hicapnonzero}
agesvec <- c('5059','6064','6569','7074','7576','7677','7778','7879','7980','80')
hicapnames <- paste('fraction_hicap_nonzero_', agesvec, sep='')
hicapnonzero = base[,c('year',hicapnames)]
ggplot(hicapnonzero, aes(x=year, ymin=0, ymax=1)) +
geom_line(aes(y=fraction_hicap_nonzero_5059, colour='50-59')) + 
geom_line(aes(y=fraction_hicap_nonzero_6064, colour='60-64')) + 
geom_line(aes(y=fraction_hicap_nonzero_6569, colour='65-69')) +
geom_line(aes(y=fraction_hicap_nonzero_7074, colour='70-74')) +
geom_line(aes(y=fraction_hicap_nonzero_7576, colour='75-79')) +
geom_line(aes(y=fraction_hicap_nonzero_7677, colour='75-79')) +
geom_line(aes(y=fraction_hicap_nonzero_7778, colour='75-79')) +
geom_line(aes(y=fraction_hicap_nonzero_7879, colour='75-79')) +
geom_line(aes(y=fraction_hicap_nonzero_7980, colour='75-79')) +
geom_line(aes(y=fraction_hicap_nonzero_80, colour='80+')) +
opts(legend.position='bottom') +
labs(y='Fraction with Nonzero Capital Income', colour='Age Group')
\end{Rcode}

\begin{Rcode}{!echo,savefig,label=exemptbyage}
agesvec <- c('6569','7074','7579','80pp')
exnames <- paste('ttl_mcare_pta_exempt', agesvec, sep='')
popnames <- paste('ttl_mcare_pta_enroll', agesvec, sep='')
exempt <- base[,c('year',exnames)]
exempt[,exnames] <- exempt[,exnames] / base[,popnames]
ggplot(exempt, aes(x=year, ymin=0, ymax=1)) +
geom_line(aes(y=ttl_mcare_pta_exempt6569, colour='65-69')) +
geom_line(aes(y=ttl_mcare_pta_exempt7074, colour='70-74')) +
geom_line(aes(y=ttl_mcare_pta_exempt7579, colour='75-79')) +
geom_line(aes(y=ttl_mcare_pta_exempt80pp, colour='80+')) +
opts(legend.position='bottom') +
labs(y='Fraction Exempt from Premium', colour='Age Group')
\end{Rcode}

\begin{Rcode}{!echo,savefig,label=poverty}
y2010 <- read.dta('../output/ACA/detailed_output/y2010_rep1.dta')
y2020 <- read.dta('../output/ACA/detailed_output/y2020_rep1.dta')
y2030 <- read.dta('../output/ACA/detailed_output/y2030_rep1.dta')
y2050 <- read.dta('../output/ACA/detailed_output/y2050_rep1.dta')
pop <- rbind(y2010, y2020, y2030, y2050)
pop <- subset(pop, ldied==0, select=c(year, age, poverty_level, weight))
pop[,'agec'] <- agegrp(pop[,'age'])
pop.split <- split(pop, list(pop[,'agec'], pop[,'year']))

plevel <- sapply(pop.split, FUN=function(x)
weighted.mean(x[,'poverty_level'], x[,'weight']))

n <- do.call(rbind, strsplit(names(plevel), '.', fixed=T))
f <- data.frame(poverty_level = plevel, agec=n[,1], year=n[,2])
ggplot(f) + 
geom_point(aes(x=agec, y=poverty_level, colour=year, ymin=0, ymax=1)) + 
opts(legend.position='bottom')
\end{Rcode}

\begin{Rcode}{!echo,savefig,label=povertyfrac}
pfrac <- sapply(pop.split, FUN=function(x) sum(subset(x, poverty_level
<= 1.38)[,'weight'])/sum(x[,'weight']))

g <- data.frame(poverty_fraction=pfrac, agec=n[,1], year=n[,2])
ggplot(g) + 
geom_point(aes(x=agec, y=poverty_fraction, colour=year, ymin=0, ymax=1)) + 
opts(legend.position='bottom')
\end{Rcode}

\begin{Rcode}{!echo,savefig,label=oldincome}
base <-  read.dta('../output/ACA/summary.dta')
med.gross <- base[,c('year','gross_mcare')]
hicap <- base[,c('year','hicap_mcare')]
hatota <- base[,c('year','hatota')]
lloghatotax <- base[,c('year','lloghatotax')]

names(med.gross) <- c('year','Value')
names(hicap) <- c('year','Value')
names(hatota) <- c('year','Value')
names(lloghatotax) <- c('year','Value')

h.plus.g <- hicap
h.plus.g[,'Value'] <- hicap[,'Value'] + med.gross[,'Value']

med.gross[,'series'] <- 'Average Gross'
hicap[,'series'] <- 'Average Capital Income'
hatota[,'series'] <- 'Average Household Wealth'
h.plus.g[,'series'] <- 'Average Gross plus Average Capital Income'
lloghatotax[,'series'] <- 'Average Lag of Transformed HaToTaX'

med.gross[,'scale'] <- 'small linear'
hicap[,'scale'] <- 'small linear'
hatota[,'scale'] <- 'large linear'
h.plus.g[,'scale'] <- 'small linear'
lloghatotax[,'scale'] <- 'log-ish'

f <- rbind(med.gross, hicap, h.plus.g, hatota, lloghatotax)
ggplot(f) + geom_line(aes(x=year, y=Value, colour=series)) +
facet_grid(scale ~ ., scales='free_y') +
opts(legend.position='bottom') + scale_colour_brewer(name='Income Source', 'Dark2')
\end{Rcode}

\begin{Rcode}{!echo,savefig,label=grossbreakdown}
ggplot(base) + 
geom_line(aes(x=year, y=ry_earn, colour='Yearly Earnings')) +
geom_line(aes(x=year, y=ssben, colour='SS Benefit')) + 
geom_line(aes(x=year, y=diben, colour='DI Benefit')) +
geom_line(aes(x=year, y=ssiben, colour='SSI Benefit')) +
geom_line(aes(x=year, y=dbpen, colour='DB Pension')) +
opts(legend.position='bottom')
\end{Rcode}

\begin{figure}[ht]
\centering
\recallfig{exemptions}
\caption{Fraction of Medicare Enrollees Below 138\% FPL}
\label{fig:exemptions}
\end{figure}

\begin{figure}[ht]
\centering
\recallfig{poverty}
\caption{Average Poverty Level by Age Group}
\label{fig:poverty}
\end{figure}

\begin{figure}[ht]
\centering
\recallfig{povertyfrac}
\caption{Fraction Below 138\% of Poverty Level by Age Group}
\label{fig:povertyfrac}
\end{figure}

\begin{figure}[ht]
\centering
\recallfig{exemptbyage}
\caption{Fraction Below 138\% of FPL by Age Group Over Time}
\end{figure}

\begin{figure}[ht]
\centering
\recallfig{oldincome}
\caption{Means for People Eligible for Medicare}
\label{fig:oldincome}
\end{figure}

\begin{figure}[ht]
\centering
\recallfig{grossbreakdown}
\caption{Mean Income Components for Medicare Eligible}
\label{fig:grossbreakdown}
\end{figure}

\begin{figure}[ht]
\centering
\recallfig{hicapnonzero}
\caption{Fraction of Population with Nonzero Capital Income}
\end{figure}

\begin{table}[ht]
\centering
\caption{Nonzero Capital Income Regression Coefficients}
\verbatiminput{../output/ACA/FEM_CPP_settings/models/hicap_nonzero.est}
\label{tab:hicapnonzero}
\end{table}

\begin{figure}[ht]
\centering
\recallfig{wlthnonzero}
\caption{Fraction of Population with Nonzero Wealth}
\end{figure}

\begin{table}[ht]
\centering
\caption{Nonzero Wealth Regression Coefficients}
\verbatiminput{../output/ACA/FEM_CPP_settings/models/wlth_nonzero.est}
\label{tab:wlthnonzero}
\end{table}

\begin{table}[ht]
\centering
\caption{Nonzero Wealth Regression Coefficients without FRBYR}
\verbatiminput{../output/noRBYR/FEM_CPP_settings/noRBYR/models/wlth_nonzero.est}
\label{tab:wlthnonzero_nofrbyr}
\end{table}

\begin{Rcode}{!echo}
covlist <- read.table('../FEM_CPP_settings/models/wlth_nonzero.est', stringsAsFactors=F, skip=2)
covlist <- covlist[1:nrow(covlist)-1,]
row.names(covlist) <- covlist[,1]
varnames <- covlist[,1]
newnames <- paste(varnames,'scaled',sep='.')
coeffs <- covlist[,2]
years <- seq(2010, 2050, 10)
dat <- lapply(years, FUN=function(y, vars, cs, newvars) {
  y <- read.dta(file.path('../output/ACA/detailed_output/', paste('y',y,'_rep1.dta', sep='')))
  y <- subset(y, ldied==0)
  c.mat <- matrix(cs, nrow=1)
  y[,newvars] <- y[,vars] * c.mat[rep.int(1,nrow(y)),]
  d <- sapply(newvars, FUN=function(v, dat) weighted.mean(dat[,v], dat[,'weight'], na.rm=T), y)
  names(d) <- newvars
  return(d)}, varnames, coeffs, newnames)
names(dat) <- as.character(years)
dat <- as.data.frame(do.call(rbind, dat))
dat[,'year'] <- as.numeric(row.names(dat))
dat.l <- as.list(dat)
print(dat)
options(warn=2)
g <- ggplot(dat)
for(i in names(dat)) {
  if(i != 'year.scaled' && i != 'year')
    g <- g + geom_line(aes_string(x='year', y=i))
}
g + opts(legend.position='right')
\end{Rcode}

\section{ROC Comparison on frbyr}

ROC curve.

\begin{Statacode}{!echo}
set mem 500M
global ghregdir "../FEM_Stata/Code"
adopath ++ "../FEM_Stata/Makedata/HRS"
adopath ++ "../FEM_Stata/Estimation"
adopath ++ "../FEM_Stata/Code"

use "/homer/c/Retire/FEM/rdata_js/hrs17r_transition.dta", clear
* FOR hacohort = 0 & 1 in wave 2 & 3 no info on SSI claiming
replace ssiclaim = -2 if inlist(hacohort, 0, 1) & inlist(wave,3,4)

replace wlth_nonzero = hatota > 0

*** CHANGE hatota, earnings VARIABLES
drop floghatota floghatotax lloghatota lloghatotax llogiearn llogiearnx flogiearn flogiearnx
set more off
foreach i in hatota hatotax iearn iearnx{
  egen flog`i' = h(f`i')
  replace flog`i' = flog`i'/100
  egen llog`i' = h(l`i')
  replace llog`i' = llog`i'/100
}


*** DEPENDENT VARIABLES
global bin_econ anyhi diclaim ssclaim dbclaim ssiclaim nhmliv work wlth_nonzero hicap_nonzero
global bin_hlth died hearte stroke cancre hibpe diabe lunge memrye
global order smkstat funcstat


*** Create obesity splines
foreach x in lobese_1 lobese_2 lobese_3 loverwt lnormwt fobese_1 fobese_2 fobese_3 foverwt fnormwt {
  cap drop `x'
}

local log_30 = log(30)
mkspline llogbmi_l30 `log_30' llogbmi_30p = llogbmi
mkspline flogbmi_l30 `log_30' flogbmi_30p = flogbmi

global bmivars llogbmi_l30 llogbmi_30p flogbmi_l30 flogbmi_30p


*** GENERATE THE AGE SPLINE VARIABLES
foreach x in lage6061 lage6263 lage64e lage6566 lage6770 lage65l lage6574 lage75l lage75p lage62e lage63e {
  cap drop `x'
}

local age_var age_iwe

gen lage6061 = floor(l`age_var') == 58 | floor(l`age_var') == 59 if l`age_var' < .
gen lage6263 = inrange(floor(l`age_var'),60,61) if l`age_var' < .
gen lage64e = floor(l`age_var') == 62 if l`age_var' < . 
gen lage6566 = floor(l`age_var') == 63 | floor(l`age_var') == 64 if l`age_var' < .
gen lage6770 = inrange(floor(l`age_var'),65,68) if l`age_var' < . 

gen lage65l  = min(63,l`age_var') if l`age_var' < .
gen lage6574 = min(max(0,l`age_var'-63),73-63) if l`age_var' < .
gen lage75l = min(l`age_var', 73) if l`age_var' < . 
gen lage75p = max(0, l`age_var'-73) if l`age_var' < . 

gen lage62e = floor(l`age_var') == 60 if l`age_var' < .
gen lage63e = floor(l`age_var') == 61 if l`age_var' < .

mkspline la6 58 la7 73 la7p = l`age_var'

gen logdeltaage = log(`age_var' - l`age_var')

*** GENERATE WEAVE DUMMIES
gen w3 = wave == 3
gen w4 = wave == 4
gen w5 = wave == 5
gen w6 = wave == 6
gen w7 = wave == 7

*** INDEPENDENT VARIABLES

*** Demographics
global dvars black hispan hsless college male 
*** Initial values
#d;
global fvars fhearte fstroke fcancre fhibpe fdiabe flunge fsmokev fsmoken fiadl1 fadl12 fadl3 
fwidowed fsingle fwork flogiearnx fwlth_nonzero floghatotax 
flogaime flogq fshlt fanydb frdb_na_2 frdb_na_3 frdb_na_4 fanydc flogdcwlthx ;
#d cr
*** values of health variables at t-1
global lvars_hlth lhearte lstroke lcancre lhibpe ldiabe llunge liadl1 ladl12 ladl3 lsmoken lwidowed 
*** values of econ variables at time t-1
global lvars_econ lwork llogiearnx lwlth_nonzero lloghatotax ldiclaim lssiclaim lssclaim ldbclaim lnhmliv

*** FOR MORTALITY
global allvars_died $dvars lage65l lage6574 lage75p $lvars_hlth $fvars

*** FOR CHRONIC CONDITIONS AND ORDINAL OUTCOMES
global allvars_hlth $dvars lage65l lage6574 lage75p $lvars_hlth  $fvars $bmivars logdeltaage

takestring, oldlist($allvars_hlth) newname("allvars_hearte") extlist("lhearte fhearte lstroke llunge lcancre liadl1 ladl12 ladl3")
takestring, oldlist($allvars_hlth) newname("allvars_stroke") extlist("lstroke fstroke lstroke llunge liadl1 ladl12 ladl3")
takestring, oldlist($allvars_hlth) newname("allvars_cancre") extlist("lhearte lstroke lcancre lhibpe ldiabe llunge fcancre liadl1 ladl12 ladl3")
takestring, oldlist($allvars_hlth) newname("allvars_hibpe")  extlist("lhearte lstroke lcancre lhibpe llunge fhibpe liadl1 ladl12 ladl3")
takestring, oldlist($allvars_hlth) newname("allvars_diabe")  extlist("lhearte lstroke lcancre lhibpe ldiabe llunge fdiabe liadl1 ladl12 ladl3")
takestring, oldlist($allvars_hlth) newname("allvars_lunge")  extlist("lhearte lstroke lcancre lhibpe ldiabe llunge flunge liadl1 ladl12 ladl3")
takestring, oldlist($allvars_hlth) newname("allvars_memrye") extlist("lhearte lstroke lcancre lhibpe ldiabe llunge flunge")

global allvars_smkstat $allvars_hlth
global allvars_funcstat $allvars_hlth

global allvars_logbmi $allvars_hlth frbyr

*** FOR ECONOMIC OUTCOMES
global allvars_econ1 $dvars lage65l lage6574 lage75p $lvars_hlth $lvars_econ $fvars logdeltaage
global allvars_econ2 $dvars lage6061 lage62e lage63e lage64e lage6566 lage6770 $lvars_hlth $lvars_econ $fvars logdeltaage
global allvars_econ3 $dvars la6 la7 la7p $lvars_hlth $lvars_econ $fvars w3 w4 w5 w6 w7 logdeltaage
global allvars_econ4 $dvars lage65l lage6574 lage75p $lvars_hlth $lvars_econ $fvars logdeltaage

takestring, oldlist($allvars_econ1) newname("allvars_anyhi")  extlist("lage75p ldbclaim lssiclaim lnhmliv lsmoken")
takestring, oldlist($allvars_econ1) newname("allvars_diclaim")  extlist("lage75p lssiclaim lssclaim ldbclaim lnhmliv lsmoken")
takestring, oldlist($allvars_econ1) newname("allvars_dbclaim")  extlist("lage75p fwork fanydb lssiclaim ldbclaim lwork lnhmliv lsmoken")
takestring, oldlist($allvars_econ1) newname("allvars_ssiclaim")  extlist("lnhmliv lsmoken")
takestring, oldlist($allvars_econ4) newname("allvars_nhmliv")  extlist("ldiclaim lssiclaim lssclaim ldbclaim lwork llogiearnx lsmoken frdb_na_4")
takestring, oldlist($allvars_econ1) newname("allvars_iearnx")  extlist("lssiclaim lnhmliv lsmoken lage75p")

takestring, oldlist($allvars_econ2) newname("allvars_ssclaim")  extlist("lssiclaim lssclaim lnhmliv lsmoken")
takestring, oldlist($allvars_econ2) newname("allvars_work")  extlist("lssiclaim lnhmliv lsmoken")

takestring, oldlist($allvars_econ3) newname("allvars_wlth_nonzero")  extlist("lssiclaim lsmoken")
takestring, oldlist($allvars_econ3) newname("allvars_hatotax")  extlist("lssiclaim lsmoken")
takestring, oldlist($allvars_econ3) newname("allvars_hicap_nonzero") extlist("lssiclaim lsmoken")
takestring, oldlist($allvars_econ3) newname("allvars_hicap") extlist("lssiclaim, lsmoken")

/*********************************************************************/
* ESTIMATE BINARY OUTCOMES
/*********************************************************************/
probit wlth_nonzero $allvars_wlth_nonzero if wlth_nonzero!=-2&wlth_nonzero!=9
lroc, nograph
probit wlth_nonzero $allvars_wlth_nonzero frbyr if wlth_nonzero!=-2&wlth_nonzero!=9
lroc, nograph

\end{Statacode}

\end{document}